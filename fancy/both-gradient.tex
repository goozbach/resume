% vim: set ts:2 et sw:2
\documentclass[svgnames]{scrartcl}
\usepackage[english]{babel}
\usepackage[]{moresize}
\usepackage{tikz}
\usetikzlibrary{fadings,patterns}
\usepackage{lipsum}
\definecolor{VeryDarkGray}{gray}{0.25}
\newsavebox{\tempbox}
\newcommand\fadetext[1]{%
  \begin{tikzfadingfrompicture}[name=fadetext]
    \node [text=white] {\normalfont \HUGE \bfseries #1};
  \end{tikzfadingfrompicture}
  \begin{lrbox}{\tempbox}%
    \begin{tikzpicture}
      \node [text=white,inner sep=0pt,outer sep=0pt] (textnode) {\normalfont \HUGE \bfseries #1};
      \shade[path fading=fadetext,fit fading=false,top color=white,bottom color=lightgray]
      (textnode.south west) rectangle (textnode.north east);
    \end{tikzpicture}%
  \end{lrbox}
  % Now we use the fading in another picture:
  {\usebox\tempbox{}}%
}
\begin{document}
\begin{tikzpicture}[remember picture,overlay]
  \node[yshift=-4cm] at (current page.north west){
    \begin{tikzpicture}[remember picture, overlay]
      \shade[top color=VeryDarkGray, bottom color=white] (0,0) rectangle (\paperwidth,4cm);
    \end{tikzpicture}
  };
\end{tikzpicture}

\fadetext{GOOZBACH}
\lipsum[1-3]
\end{document}
